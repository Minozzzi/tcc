%%%% RESUMO
%%
%% Apresentação concisa dos pontos relevantes de um texto, fornecendo uma visão rápida e clara do conteúdo e das conclusões do
%% trabalho.

\begin{resumoutfpr}%% Ambiente resumoutfpr
  O estado do Paraná, como um dos principais contribuintes para a produção nacional de leite, enfrenta desafios na gestão eficaz do rebanho, especialmente no manejo alimentar. Apesar da qualidade garantida da alimentação, muitas vezes, ela não atende às necessidades nutricionais do gado leiteiro. Nesse contexto, propõe-se um sistema web para otimizar o tratamento e armazenamento de dados relacionados ao rebanho e alimentação. O objetivo central do sistema é aprimorar a eficiência na gestão do rebanho, proporcionando uma abordagem mais eficaz para a alimentação e nutrição do gado, visando melhoras significativas na produção leiteira. Destinado a auxiliar técnicos do \gls{IDR-PR} e produtores, o sistema abrange a gestão dos animais, nutrição, acompanhamento nas propriedades rurais e informações das propriedades. A falta de manejo adequado do rebanho, apesar dos avanços tecnológicos, é evidente, e um sistema robusto pode oferecer uma solução integrada. A metodologia ágil adotada no desenvolvimento do projeto permitirá a adaptação contínua às mudanças de requisitos, proporcionando melhorias com base em \textit{feedbacks} contínuos ao longo dos marcos do projeto. Ao abordar especificamente as necessidades nutricionais dos animais, o sistema busca não apenas armazenar dados, mas também proporcionar uma compreensão aprofundada das práticas atuais dos técnicos do \gls{IDR-PR}, facilitando a tomada de decisões informadas. A sincronização eficiente dos dados entre o cliente web e a API é crucial, considerando os desafios potenciais de conectividade em ambientes rurais.
\end{resumoutfpr}
