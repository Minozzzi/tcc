%%%% ABSTRACT
%%
%% Versão do resumo para idioma de divulgação internacional.

\begin{abstractutfpr}%% Ambiente abstractutfpr
  The state of Paraná, as one of the main contributors to national milk production, faces challenges in the effective management of the herd, especially in feeding management. Despite the guaranteed quality of the feed, it often does not meet the nutritional needs of dairy cattle. In this context, a web system is proposed to optimize the treatment and storage of data related to the herd and feeding. The central goal of the system is to enhance efficiency in herd management, providing a more effective approach to cattle feeding and nutrition, aiming for significant improvements in dairy production. Intended to assist technicians from \gls{IDR-PR} and producers, the system covers animal management, nutrition, on-farm monitoring, and property information. The lack of proper herd management, despite technological advances, is evident, and a robust system can offer an integrated solution. The agile methodology adopted in the project's development allows continuous adaptation to changing requirements, providing improvements based on continuous feedback throughout the project milestones. By specifically addressing the nutritional needs of the animals, the system seeks not only to store data but also to provide an in-depth understanding of the current practices of \gls{IDR-PR} technicians, facilitating informed decision-making. Efficient data synchronization between the web client and the API is crucial, considering potential connectivity challenges in rural environments.
\end{abstractutfpr}
