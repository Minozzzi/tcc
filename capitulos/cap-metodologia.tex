%%%% CAPÍTULO 3 - MATERIAL E MÉTODOS (PODE SER OUTRO TÍTULO DE ACORDO COM O TRABALHO REALIZADO)

\chapter{Materiais e Método}\label{cap:materialemetodos}

Este capítulo está estruturado em duas partes iniciais. A primeira aborda os elementos relacionados às ferramentas e tecnologias, enquanto a segunda se concentra na descrição do método, que compreende a sequência das principais atividades realizadas para a criação e evolução do sistema. Sendo a principal ferramenta utilizada para o desenvolvimento do projeto, a biblioteca React.

\section{Materiais}\label{sec:materiais}

Esta seção fornece uma breve visão geral das principais ferramentas empregadas no desenvolvimento do sistema proposto.

\begin{tabframed}[htb]
  \caption{Lista de ferramentas e tecnologias}
  \label{quad:materiais}
  \renewcommand{\arraystretch}{2}
  \begin{tabular}{|l|l|l|l|}
    \cline{1-4}
    \textbf{Ferramenta/Tecnologia}    &
    \textbf{Versão}                   &
    \textbf{Disponível em}            &
    \textbf{Finalidade}
    \\ \cline{1-4}

    \gls{VSCode}                      &
    latest                            &
    https://code.visualstudio.com/    &
    Editor de código-fonte
    \\ \cline{1-4}

    TypeScript                        &
    latest                            &
    https://www.typescriptlang.org/   &
    Lingugagem de programação
    \\ \cline{1-4}

    React                             &
    18.x.x                            &
    https://react.dev/                &
    \multicolumn{1}{|p{4cm}|}{\raggedright Biblioteca para construir \gls{UI} interativas e reativas em aplicações web}
    \\ \cline{1-4}

    Linaria                           &
    5.x.x                             &
    https://linaria.dev/              &
    \multicolumn{1}{|p{4cm}|}{\raggedright Biblioteca para estilização \textit{css-in-js}}
    \\ \cline{1-4}

    Storybook                         &
    7.x.x                             &
    https://storybook.js.org/         &
    \multicolumn{1}{|p{4cm}|}{\raggedright Documentar, testar e visualizar componentes de \gls{UI}}
    \\ \cline{1-4}

    LucidChart                        &
                                      &
    https://www.lucidchart.com/pages/ &
    \multicolumn{1}{|p{4cm}|}{\raggedright Ferramenta para criação de diagramas}
    \\ \cline{1-4}

    Figma                             &
                                      &
    https://www.figma.com/            &
    \multicolumn{1}{|p{4cm}|}{\raggedright Ferramenta para criação de protótipos de interfaces}
    \\ \cline{1-4}

    \multicolumn{1}{|p{4cm}|}{\raggedright Planilhas eletrônicas disponibilizadas pelo \gls{IDR-PR}} 
                                      &
                                      &
                                      &
    \multicolumn{1}{|p{4cm}|}{\raggedright Estudo do projeto e uso para levamento de requisitos}
    \\ \cline{1-4}
  \end{tabular}
  \fonte{}%% Fonte
\end{tabframed}

\subsection{Visual Studio Code}\label{subsec:vscode}

O \gls{VSCode}, é um popular \gls{IDE} desenvolvido pela Microsoft. Ele é conhecido por sua extrema flexibilidade, leveza e suporte a uma ampla variedade de linguagens de programação. O \gls{VSCode} oferece uma interface de usuário limpa e altamente personalizável, o que o torna uma escolha preferida para muitos desenvolvedores. Uma das características distintivas do \gls{VSCode} é a grande variedade de extensões disponíveis, que permitem aos desenvolvedores personalizar o ambiente de acordo com suas necessidades. Essas extensões abrangem desde suporte a linguagens específicas até ferramentas de depuração, controle de versão e integração com sistemas de gerenciamento de pacotes. Além disso, o \gls{VSCode} oferece uma experiência de depuração de primeira classe, tornando-o ideal para o desenvolvimento de aplicativos complexos \cite{Sole:2019:VisualStudioCodeDistilled}.

\subsection{TypeScript}\label{subsec:typescript}

O TypeScript é uma linguagem de programação de código aberto desenvolvida pela Microsoft, projetada para ser uma extensão do JavaScript. O que a torna única é a adição de tipagem estática, permitindo que os desenvolvedores definam tipos de dados para variáveis, parâmetros e retornos de funções. Essa tipagem estática ajuda a detectar erros em tempo de compilação, tornando o código mais seguro e menos propenso a bugs \cite{Bierman:2014:UnderstandingTypeScript}.

\subsection{React}\label{subsec:react}

O React é uma biblioteca de \gls{UI} criada pela Meta com o propósito de simplificar a construção de componentes reutilizáveis, que podem ser interativos e possuir estados \cite{Kumar:2016:ComparativeAnalysisAngularJSReactJS}.

Viabiliza o desenvolvimento de aplicativos \textit{web} de grande porte e complexidade, capacitando-os a efetuar alterações nos dados sem a necessidade de atualizar páginas completas. Isso implica que a arquitetura central do React desempenha o papel de "V" (\textit{View}) no padrão de design \gls{MVC} \cite{Aggarwal:2018:ModernWebDevelopmentUsingReactJS}. De acordo com \citeonline{Vipul:2016:ReactJSByExampleBuildingModernWebApplications}, o React elabora abstrações das visualizações ao dividir essas visualizações em componentes distintos.

Além disso, o React se baseia na premissa de que a manipulação do \gls{DOM} é uma tarefa dispendiosa que precisa ser minimizada. Além disso, reconhece que otimizar manualmente a manipulação do \gls{DOM} resultaria em uma quantidade considerável de código repetitivo e propenso a erros. O React aborda esse problema disponibilizando aos desenvolvedores um \gls{DOM} virtual para renderização em vez do \gls{DOM} real. Ele identifica as discrepâncias entre o \gls{DOM} real e o \gls{DOM} virtual, executando o mínimo de operações do \gls{DOM} necessárias para alcançar o novo estado \cite{Vipul:2016:ReactJSByExampleBuildingModernWebApplications}.

\section{Método}\label{sec:metodo}

Para o desenvolvimento deste projeto, será adotada uma abordagem ágil, conforme sugerido por \citeonline{Pontes:2019:MetodologiasAgeisDesenvolvimentoSoftwares}. A metodologia ágil permitirá a coleta contínua de feedbacks ao longo das sprints e marcos do projeto, visando efetuar melhorias com base nesses retornos.
Além disso, a metodologia ágil permitirá a adaptação do projeto às mudanças de requisitos e necessidades do cliente, que poderão ocorrem durante o desenvolvimento.

Inicialmente, para a coleta dos requisitos, foram feitas análises minuciosas das informações fornecidas pelos técnicos do \gls{IDR-PR}, que desempenham um papel fundamental na área em questão. Além disso, foi examinado de perto os documentos disponibilizados por esses profissionais, buscando uma compreensão abrangente das necessidades e desafios presentes. Através dessa abordagem, foi permitido obter uma visão completa e embasada para que o projeto possa avançar. Deste modo, possibilitando profundo entendimento dos procedimentos corriqueiros indispensáveis do sistema e das dificuldades enfrentadas pelos técnicos.

Através da utilização de aplicações de criação de planilhas eletrônicas, os técnicos do \gls{IDR-PR} realizam o gerenciamento dos dados das propriedades que estão sob sua supervisão. Essas planilhas fornecem uma base sólida para análisar e coletar os dados necessários para o sistema, bem como compreender a interação entre eles.

Após entendimento das funcionalidades e de como os dados se comportam através dos documentos fornecidos, foi possível o levantamento dos requisitos funcionais e não funcionais do sistema, tal como a diagramação dos casos de uso.

Na etapa subsequente, procederá à prototipação do sistema, incluindo suas interfaces, com o propósito de aprimorar a compreensão do que será desenvolvido. Isso irá possibilitar a eliminação de elementos desnecessários e a realização de testes para avaliar a experiência do usuário no sistema.

A seguir, o processo de desenvolvimento dos códigos-fonte da aplicação será iniciado. Simultaneamente, os testes do sistema serão conduzidos com o intuito de reduzir e corrigir eventuais erros durante a etapa de desenvolvimento.

