%%%% CAPÍTULO 2 - REVISÃO DA LITERATURA (OU REVISÃO BIBLIOGRÁFICA, ESTADO DA ARTE, ESTADO DO CONHECIMENTO)
%%
%% O autor deve registrar seu conhecimento sobre a literatura básica do assunto, discutindo e comentando a informação já publicada.
%% A revisão deve ser apresentada, preferencialmente, em ordem cronológica e por blocos de assunto, procurando mostrar a evolução do tema.
%% Título e rótulo de capítulo (rótulos não devem conter caracteres especiais, acentuados ou cedilha)
\chapter{Referencial te\'orico}\label{cap:referencialTeorico}

Este capítulo explora a fundamentação teórica deste trabalho, cujo conteúdo explana sobre a alimentação nutritiva para o gado leiteiro.

\section{Alimentação Balanceada para o Gado Leiteiro}\label{sec:alimentacaoBalanceadaGadoLeiteiro}

A demanda por nutrientes é substancialmente elevada em animais em lactação, independentemente de serem criados a pasto ou em sistemas de confinamento. Nestas circunstâncias, prever com precisão o consumo alimentar é de extrema relevância para garantir a eficiência no sistema de produção \cite{Kolln:2014:AvaliacaoSistemaOtimizacaoRacaoVacaLeiteira}.

Em contextos de produção de leite, a nutrição dos animais desempenha um papel fundamental na busca por maior eficiência e qualidade, ao mesmo tempo em que se busca reduzir os custos envolvidos, como destacado por \citeonline{Tomich:2015:NutricaoPrecisaoPecuariaLeiteira}. Essa ênfase na nutrição é essencial para otimizar a produtividade e a qualidade do produto gerado na atividade leiteira.

Especialmente na pecuária leiteira, o \gls{CMS} desempenha um papel crucial, exercendo influência direta sobre fatores relacionados à produtividade. É fundamental alcançar níveis destacados tanto em reprodução quanto em produção de leite, como destacado por \citeonline{Zanin:2017:EquacoesEstimarConsumoVacasLeiteiras}.

O \gls{CMS}, conforme indicado por \citeonline{Mertens:1987:PredictingIntakeDigestibility}, desempenha um papel significativo no desempenho dos bovinos, representando potencialmente entre 60\% a 90\% das variações observadas. Isso significa que fatores relacionados ao \gls{CMS}, como a quantidade e qualidade da matéria seca consumida pelos bovinos, têm uma influência substancial no desempenho do gado. No entanto, os outros 10\% a 40\% das variações no desempenho estão relacionados a outros fatores, como a qualidade nutricional dos alimentos fornecidos aos bovinos. Portanto, a gestão adequada do \gls{CMS} é essencial para otimizar o desempenho do gado, mas não se pode negligenciar a importância de fornecer alimentos nutricionalmente equilibrados.

Dessa forma, há vários modelos matemáticos que visam predizer o \gls{CMS}, para que seja possível torna a produção mais sustentável e escalável. Contudo, a opção de uma dieta de menor custo nem sempre reverte em maior lucratividade para o produtor, pois ao fazer incrementos pequenos no custo podem gerar um aumento significativo no desempenho \cite{Kolln:2014:AvaliacaoSistemaOtimizacaoRacaoVacaLeiteira}.

Quando se trata da atividade leiteira, existem dois modelos que se destacam, o \gls{CNCPS} e \gls{NRC}, ambos norte americanos. Enquanto o \gls{CNCPS} determina as necessidades com base no peso do animal e na produção de leite, o \gls{NRC} considera adicionalmente informações relacionadas à composição do leite e à fase da lactação \cite{Zanin:2017:EquacoesEstimarConsumoVacasLeiteiras}.

Apesar do modelo \gls{CNCPS} ser norte americano, empregando informações de regiões com clima temperado e nutrição caracteristíca do sistema leiteiro norte-americano, o mesmo foi adaptado e validado para alimentos e animais presentes em condições tropicais \cite{Lanna:1999:ModelosLinearesNaoLinearesUsoNutrientes}.

De acordo com \citeonline{Arrigoni:2023:NiveisElevadosConcentrado} a natureza dos alimentos, seja ela caracterizada como volumosa ou concentrada, e os níveis de nutrientes contidos nesses alimentos desempenham um papel crucial na influência do comportamento alimentar dos animais. Esses fatores exercem um impacto significativo no desempenho e na produtividade dos animais.

No processo de balanceamento nutricional para animais, conforme estabelecido por \citeonline{Salman:2011:ManualFormulacaoRacaoVacasLeiteiras}, diversas etapas precisam ser cuidadosamente seguidas para assegurar a eficácia e a saúde do rebanho. Estas diretrizes fornecem um arcabouço fundamental para garantir que as necessidades nutricionais dos animais sejam atendidas de forma precisa e sustentável, sendo um elemento essencial na promoção do bem-estar e na maximização da produtividade. A seguir, serão apresentadas as principais etapas desse processo, proporcionando uma visão geral das práticas que sustentam a nutrição adequada para animais de produção.

\begin{enumerate}
  \item Inicialmente, é essencial realizar a identificação dos animais que serão alvo do balanceamento da ração.

  \item Em seguida, é necessário estabelecer as necessidades nutricionais desses animais com base nas características previamente identificadas.

  \item Um passo crucial envolve a coleta e quantificação dos alimentos disponíveis, levando em conta a disponibilidade e qualidade dos recursos alimentares.

  \item Para garantir um balanceamento preciso, é fundamental relacionar a composição química e o valor energético dos alimentos a serem utilizados, considerando os nutrientes de interesse.

  \item Posteriormente, a ração é balanceada, priorizando a proteína bruta e a energia, visando atender às necessidades nutricionais estabelecidas.

  \item Uma vez que o cálculo da ração está concluído, é imperativo realizar uma verificação minuciosa para assegurar que todas as exigências nutricionais dos animais tenham sido devidamente atendidas.
\end{enumerate}

Com o objetivo de satisfazer as necessidades dos animais, torna-se imprescindível a seleção de alimentos com base em seu valor nutricional. Para alcançar esse propósito, é fundamental que cada tipo de alimento apresente sua composição química, o que determinará a quantidade a ser considerada durante o processo de balanceamento \cite{Tomich:2015:NutricaoPrecisaoPecuariaLeiteira}.

Visando atender às necessidades nutricionais dos animais, foram definidos métodos práticos para formulação de rações \cite{Salman2020:ManualFormulacaoRacaoVacasLeiteiras}. Sendo eles, o método algébrico e o método do quadrado de Pearson.

O método algébrico, viabiliza a fusão de dois ou mais componentes e envolve a formulação de um sistema de equações simultâneas, onde as incógnitas correspondem aos componentes a serem incorporados na ração. A complexidade desse método aumenta de maneira gradual à medida que se incluem um maior número de componentes e nutrientes no cálculo.

Um exemplo prático desse método pode ser ilustrado ao considerar uma ração concentrada com 18\% de \gls{PB}, composta por farelo de algodão (representado como X) e grãos de milho (representados como Y) em uma quantidade total de 100 kg, valores baseados em \citeonline{Salman:2011:ManualFormulacaoRacaoVacasLeiteiras}. A equação que modela essa situação é a seguinte:

\begin{equation}
  \label{eq:algebrico1}
  X + Y = 100
\end{equation}

Isolando a variável Y, obtemos:
\begin{equation}
  \label{eq:algebrico2}
  Y = 100 - X
\end{equation}

Substituindo a \seqref{eq:algebrico2} na \seqref{eq:algebrico1}, obtém-se:
\begin{equation}
  \label{eq:algebrico3}
  X + (100 - X) = 100
\end{equation}

Ao incorporar os teores de \gls{PB} da ração (18\%), do farelo de algodão (35,65\%) e dos grãos de milho (14,5\%) na \seqref{eq:algebrico3}, tem-se:
\begin{equation}
  \label{eq:algebrico4}
  35,65 (X) + 14,5 (100 - X) = 18 (100)
\end{equation}

Onde:
\begin{equation}
  \label{eq:algebrico5}
  35,65X + 1450 - 14,5X = 1800
\end{equation}
\begin{equation}
  \label{eq:algebrico6}
  21,15X = 350
\end{equation}
\begin{equation}
  \label{eq:algebrico7}
  X = 350 / 21,54 = 16,55\%
\end{equation}

Substituindo o valor de X na \seqref{eq:algebrico2}, encontramos a quantidade de milho (Y) na ração:
\begin{equation}
  \label{eq:algebrico8}
  Y = 100 - 16,55 = 83,45\%
\end{equation}

Portanto, a ração será constituída de 16,55\% de farelo de algodão e 16,55\% de grão de milho.

O Método do Quadrado de Pearson, é de natureza direta e permite a determinação das proporções de dois componentes em uma mistura, com o intuito de atingir um nível de nutriente específico, geralmente a proteína. Este método viabiliza o uso de dois alimentos ou conjuntos de alimentos que tenham sido previamente mesclados.

Um exemplo prático que ilustra este método é o balanceamento de uma ração concentrada contendo 18\% de \gls{PB} e 80\% de \gls{NDT} utilizando o método do Quadrado de Pearson. Neste método, consideramos os teores de proteína dos ingredientes disponíveis, como o grão de milho moído e o farelo de soja, com teores de 9,82\% e 47,64\% de \gls{PB}, respectivamente, valores baseados em \citeonline{Salman:2011:ManualFormulacaoRacaoVacasLeiteiras}.

O processo inicia-se pela construção de um esquema quadrado, onde o valor do teor de \gls{PB} da mistura é colocado no centro do quadrado. À esquerda, são inseridos os teores de \gls{PB} dos dois ingredientes da mistura, enquanto à direita são registradas as diferenças numéricas entre os valores dos ingredientes e o teor de \gls{PB} da mistura, sendo: 
\begin{equation}
  \label{eq:pearson1}
  18 - 9,82 = 8,18
\end{equation}

\begin{equation}
  \label{eq:pearson2}
  47,64 - 18 = 29,64
\end{equation}

Dessa forma, para 37,82 kg da mistura, será necessário utilizar 29,64 kg de milho e 8,18 kg de farelo de soja. Portanto, para 100 kg da mistura, serão necessários 78,37 kg de milho:
\begin{equation}
 \label{eq:pearson3}
 (29,64 / 37,82)*100 
\end{equation}

E 8,18 kg de farelo de soja:
\begin{equation}
 \label{eq:pearson4}
 (8,18 / 37,82)*100
\end{equation}

Diante da extensiva quantidade de dados que necessita ser processada para alcançar um resultado, ressalta-se a crucial relevância de um sistema de informação que seja capaz de armazenar e analisar os registros, proporcionando eficiência e celeridade no desempenho das tarefas.

