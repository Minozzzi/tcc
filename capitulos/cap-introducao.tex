%%%% CAPÍTULO 1 - INTRODUÇÃO
%%
%% Deve apresentar uma visão global da pesquisa, incluindo: breve histórico, importância e justificativa da escolha do tema,
%% delimitações do assunto, formulação de hipóteses e objetivos da pesquisa e estrutura do trabalho.

%% Título e rótulo de capítulo (rótulos não devem conter caracteres especiais, acentuados ou cedilha)
\chapter{Introdução}\label{cap:introducao}

Um texto curto apresentando o capítulo.

\caixa{Atenção}{Para utilizar esse template é obrigatória a leitura do conteúdo do arquivo \texttt{readme.md}, que está neste projeto!}

\section{Considerações iniciais}\label{sec:consideracoesIniciais}

As considerações iniciais compõem um texto curto e geral apresentando uma visão geral e sucinta do assunto principal relacionado ao trabalho e a inserção do objeto de pesquisa nesse assunto \cite{Moore:2000:CMC:333067.333074}.

Em relação ao assunto, o apresentado nesta seção pode estar relacionado a trabalhos de outros autores ou ao assunto que fornece a fundamentação (motivação) para o trabalho a ser desenvolvido. Se o assunto está relacionado a trabalhos de outros autores, a contribuição do trabalho é definida em relação ao que já foi pesquisado nesse assunto. Se o assunto será utilizado para embasamento do que será proposto, explicitar como o trabalho se insere nesse assunto. A contribuição pode, ainda, estar relacionada a uma necessidade de mercado ou a uma oportunidade decorrente de algum problema real para o qual se pretender propor uma solução. Nesse caso, o assunto fornece um contexto teórico de suporte para o problema e/ou a solução.

O importante nesta seção é deixar claro do que se trata o trabalho (assunto ou tema), identificar o objeto de pesquisa, como será encaminhada a solução (procedimento metodológico, tecnologias, ferramentas utilizadas) e o que se pretende ao final do trabalho, sem explicitar a solução e os resultados.

\caixa{Atenção}{As seções a seguir são sugestões, converse com o seu orientador para ver quais seções devem ter em seu trabalho!}

\section{Objetivos}\label{sec:objetivos}

Nesta seção, serão apresentados o objetivo geral e os objetivos específicos do cliente proposto neste trabalho. O objetivo geral representa o resultado central que espera ser alcançado, enquanto os objetivos específicos delineiam as principais funcionalidades do cliente web em questão.

\subsection{Objetivo geral}\label{subsec:objetivoGeral}

Desenvolver um cliente web para controle nutricional e gerenciamento do gado leiteiro nas propriedades rurais.

\subsection{Objetivos específicos}\label{subsec:objetivosEspecificos}

\begin{itemize}
  \item Facilitar a coleta de dados do gado leiteiro da propriedade.

  \item Viabilizar o registro completo de informações das propriedades.

  \item Possibilitar o registro e identificação de doenças e pragas nas plantações da propriedade.

  \item Proporcionar o controle financeiro das propriedades, incluindo o acompanhamento de receitas e despesas.

  \item Permitir o controle detalhado de insumos e produtos utilizados nas propriedades.

  \item Simplificar o registro do fluxo de visitas à propriedade, permitindo que o técnico responsável colete dados relacionados ao gado.

  \item Habilitar a gestão dos dados de forma offline.

  \item Controlar os níveis de acesso aos dados.

  \item Assegurar a manutenção e atualização dos dados coletados das propriedades.
\end{itemize}

\section{Justificativa}\label{sec:justificativa}

Justificar o objeto de pesquisa (o que será feito) e a forma de resolução do problema (como fazer). A forma de resolução pode estar centrada no método, nas tecnologias, no uso de conceitos (fundamentação teórica).

A Justificativa explicita porque desenvolver o referido trabalho, como o mesmo se insere no contexto de pesquisa, de produção científica. Pode incluir o porquê utilizar as tecnologias e ferramentas indicadas, a contribuição em termos de inovação ou mesmo de aprendizado.

O trabalho não precisa ser justificado em decorrência de ser inovador ou por ter gerado uma significativa contribuição ao conhecimento na área em que o mesmo se insere. Pode referir-se simplesmente à aplicabilidade de conhecimentos adquiridos durante o curso. Sendo assim, a justificativa não deve ser elaborada considerando um mercado a ser atingido e sim com relação ao uso de tecnologias aprendidas e/ou estudadas, o conhecimento e aprendizado do aluno e a aplicabilidade do trabalho desenvolvido.

\section{Estrutura do trabalho}\label{sec:estruturaTrabalho}

A estrutura do trabalho contém uma relação dos capítulos e uma descrição sucinta do que cada um deles contém. Esta seção fornece uma visão geral do trabalho no sentido da sua estrutura em capítulos\footnote{Teste de nota de rodapé 2.}.

\caixa{Atenção}{O OverLeaf está demorando muito para compilar o modelo com o Capítulo de Exemplos, que explica como usar o LaTeX. Assim, esse capítulo foi removido (está comentado para não compilar), mas há um arquivo chamado \texttt{exemploPDF.pdf}, na raiz do projeto, que contém esse capítulo de exemplos!}
