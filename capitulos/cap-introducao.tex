%%%% CAPÍTULO 1 - INTRODUÇÃO
%%
%% Deve apresentar uma visão global da pesquisa, incluindo: breve histórico, importância e justificativa da escolha do tema,
%% delimitações do assunto, formulação de hipóteses e objetivos da pesquisa e estrutura do trabalho.

%% Título e rótulo de capítulo (rótulos não devem conter caracteres especiais, acentuados ou cedilha)
\chapter{Introdução}\label{cap:introducao}
De acordo com informações fornecidas pela \gls{CNA}, o setor do agronegócio registrou um crescimento de 8,36\% em 2021, alcançando uma parcela de 27,4\% no \gls{PIB} do país. Essa porcentagem representa a maior participação desde 2004, quando atingiu 27,53\%, apesar de ter ficado abaixo da estimativa anterior de 9,37\% \cite{CNA:2021:PesquisaPecuariaMunicipal2020}.

No ano de 2021, a produção de leite no Brasil superou a marca dos 35 bilhões de litros. As regiões Sul e Sudeste despontaram como as principais produtoras, contribuindo para um marco histórico em termos de valores monetários gerados, como indicado pelo \gls{IBGE} \cite{IBGE:2021:ProducaoAgropecuaria}.

Apesar da queda de 4,22\% em 2022 no \gls{PIB} brasileiro do setor do agronegócio conforme dados do \gls{CEPEA} em parceria com a \gls{CNA}, o agronegócio é um dos principais responsáveis pelo \gls{PIB}, representando cerca de 24,8\% com crescimento da pecuária em 2,11\% \cite{CEPEA:2023:PIBAgronegocio}.

De acordo com informações do Ministério da Agricultura, a agricultura familiar é um setor fundamental, empregando mais de 10 milhões de indivíduos \cite{MDA:2019:AgriculturaFamiliar}, e desempenha um papel central na produção e no cultivo dos alimentos consumidos pela população brasileira. Entre esses alimentos, destaca-se a produção de leite, que possui relevância tanto econômica quanto social, e é uma atividade comum em diversas propriedades com mão de obra familiar.

Devido aos progressos tecnológicos, é notável que o segmento de produção de leite está passando por uma intensa modernização. Isso ocorre para se adaptar às novas opções oferecidas pelos sistemas de produção, com o objetivo de aprimorar a gestão da propriedade, otimizar os processos temporais e de produção, bem como elevar a qualidade dos produtos. Esse esforço contribui, por conseguinte, para o aprimoramento da qualidade de vida no ambiente rural \cite{Botega:JVL:2007:DiagnosticoAutomacaoProducaoLeiteira}.

Contudo, o principal desafio no âmbito da produção de leite reside em proporcionar aos animais uma alimentação apropriada e de qualidade, que permita alcançar o máximo de sua capacidade de produção leiteira. Isso deve ocorrer simultaneamente à contenção dos custos associados à produção desses alimentos, a fim de não prejudicar a lucratividade da atividade. Entretanto, frequentemente ocorre que a alimentação fornecida não atende às necessidades nutricionais do animal, resultando em excessos, déficits ou inadequações nos nutrientes oferecidos. Para enfrentar esse desafio, é possível utilizar fórmulas matemáticas e modelos de otimização que permitem a formulação precisa de dietas balanceadas garantindo a nutrição ideal dos animais ao mesmo tempo que se minimizam os custos associados à produção de alimentos.

Conforme apontado por \citeonline{Vilela:2016:PecuariaLeiteBrasil}, a escassez de formação educacional tecnológica entre esses produtores se apresenta como um obstáculo considerável na adoção de práticas como o registro de receitas e despesas, além do controle zootécnico. Tal situação, por sua vez, dificulta até mesmo a utilização de ferramentas simples para a coleta de informações.

Com o objetivo de aprimorar e resolver essa questão nas propriedades rurais do Paraná, o \gls{IDR-PR} tem se empenhado em coletar informações por meio de visitas às propriedades de pequenos agricultores, com a finalidade de gerenciar e monitorar o bem-estar dos animais. Esse processo envolve a coleta de dados variados, incluindo produção de leite, gestação, nascimento, peso, quantidade e tipos de alimentos fornecidos, entre outras informações relevantes, que são posteriormente registradas em planilhas para posterior análise.

Aprimorar a coleta, armazenamento e administração de dados nessa atividade pode ser alcançado por meio da adoção de um sistema de informação. Isso permitirá que técnicos e proprietários efetuem um gerenciamento mais eficaz do rebanho, contribuindo para a tomada de decisões embasadas. Com o intuito de otimizar o uso do tempo e simplificar a carga de trabalho, esta pesquisa propõe o desenvolvimento de um sistema web, destinado a facilitar a gestão dos animais, o acompanhamento nutricional e o monitoramento do gado leiteiro nas propriedades rurais. Essa ferramenta visa beneficiar tanto os técnicos do \gls{IDR-PR} quanto os próprios produtores.

O propósito do cliente web elaborado neste estudo é operar de forma integrada com um aplicativo móvel e uma \gls{API} com arquitetura de \gls{REST}, os quais fazem parte do mesmo projeto global. Apesar de esses sistemas adicionais estarem sendo desenvolvidos concomitantemente com o presente trabalho, eles estão fora do âmbito que está sendo abordado neste contexto. Contudo, é relevante mencionar a existência desses sistemas complementares, dado que o funcionamento do sistema web será dependente da interação com a \gls{API} \gls{REST} para acessar os dados necessários e compartilhá-los com o aplicativo móvel.

Está em progresso o desenvolvimento da \gls{API} \gls{REST}, a qual será equipada com \textit{endpoints} que possibilitarão ao cliente web acessar e manipular os dados essenciais para o seu funcionamento. Essa \gls{API} será estruturada em conformidade com os princípios do estilo arquitetural \gls{REST}, promovendo, assim, uma comunicação eficaz e padronizada entre o aplicativo móvel e o cliente web. A concepção do aplicativo móvel, por sua vez, está focada na criação de uma interface simples e intuitiva, visando a facilitação de inserção e manuseio dos dados quando o técnico estiver fazendo a visita à propriedade, o aplicativo móvel funcionará de maneira complementar ao cliente web.

\section{Objetivos}\label{sec:objetivos}

Nesta seção, serão apresentados o objetivo geral e os objetivos específicos do cliente proposto neste trabalho. O objetivo geral representa o resultado central que espera ser alcançado, enquanto os objetivos específicos delineiam as principais funcionalidades do cliente web em questão.

\subsection{Objetivo geral}\label{subsec:objetivoGeral}

Desenvolver um cliente web para controle nutricional e gerenciamento do gado leiteiro nas propriedades rurais.

\subsection{Objetivos específicos}\label{subsec:objetivosEspecificos}

\begin{itemize}
  \item Facilitar a coleta de dados do gado leiteiro da propriedade.

  \item Viabilizar o registro completo de informações das propriedades.

  \item Possibilitar o registro e identificação de doenças e pragas nas plantações da propriedade.

  \item Proporcionar o controle financeiro das propriedades, incluindo o acompanhamento de receitas e despesas.

  \item Permitir o controle detalhado de insumos e produtos utilizados nas propriedades.

  \item Simplificar o registro do fluxo de visitas à propriedade, permitindo que o técnico responsável colete dados relacionados ao gado.

  \item Habilitar a gestão dos dados de forma offline.

  \item Controlar os níveis de acesso aos dados.

  \item Assegurar a manutenção e atualização dos dados coletados das propriedades.
\end{itemize}

\section{Justificativa}\label{sec:justificativa}

Justificar o objeto de pesquisa (o que será feito) e a forma de resolução do problema (como fazer). A forma de resolução pode estar centrada no método, nas tecnologias, no uso de conceitos (fundamentação teórica).

A Justificativa explicita porque desenvolver o referido trabalho, como o mesmo se insere no contexto de pesquisa, de produção científica. Pode incluir o porquê utilizar as tecnologias e ferramentas indicadas, a contribuição em termos de inovação ou mesmo de aprendizado.

O trabalho não precisa ser justificado em decorrência de ser inovador ou por ter gerado uma significativa contribuição ao conhecimento na área em que o mesmo se insere. Pode referir-se simplesmente à aplicabilidade de conhecimentos adquiridos durante o curso. Sendo assim, a justificativa não deve ser elaborada considerando um mercado a ser atingido e sim com relação ao uso de tecnologias aprendidas e/ou estudadas, o conhecimento e aprendizado do aluno e a aplicabilidade do trabalho desenvolvido.

\section{Estrutura do trabalho}\label{sec:estruturaTrabalho}

A estrutura do trabalho contém uma relação dos capítulos e uma descrição sucinta do que cada um deles contém. Esta seção fornece uma visão geral do trabalho no sentido da sua estrutura em capítulos\footnote{Teste de nota de rodapé 2.}.

\caixa{Atenção}{O OverLeaf está demorando muito para compilar o modelo com o Capítulo de Exemplos, que explica como usar o LaTeX. Assim, esse capítulo foi removido (está comentado para não compilar), mas há um arquivo chamado \texttt{exemploPDF.pdf}, na raiz do projeto, que contém esse capítulo de exemplos!}
